% definição tipo de documento
\documentclass[
12pt,       % tamanho da fonte
openright,  % inicio documento a direita
oneside,    % frente e verso
a4paper,    % folha a4
sumario = abnt-6027-2012, % Formatação do sumário: tradicional (estilo tradicional) ou abnt-6027-2012 (norma ABNT 6027-2012)
chapter = TITLE,          % Títulos de capítulos em maiúsculas 
%section = TITLE,
pretextualoneside,        % Impressão dos elementos pré-textuais pretextualoneside em um lado da folha
fontetimes,               % Fonte do texto: (times)
% fontearial,                % Fonte do texto: (arial) 
% fontecourier            % Fonte do texto: (courier), 
% fontemodern,              % Fonte do texto: (lmodern - default latex)
semrecuonosumario,        % Remoção do recuo dos itens no sumário (comente para adição do recuo, se estilo tradicional)
legendascentralizadas,    % Alinhamento das legendas centralizado (comente para alinhamento à esquerda)
pardeassinaturas,         % Assinaturas na folha de aprovação em até duas colunas (comente para em uma única coluna)
linhasdeassinaturas,      % Linhas de assinaturas na folha de aprovação (comente para remover as linhas)
english,french,spanish,
brazil      % o ultimo idioma é o padrão do documento
]{utfprct}
% -----------------------------------------------------
% pacotes utilizados no documento
\usepackage[utf8]{inputenc} % codificação texto
\usepackage[T1]{fontenc}    % codificação para adicionar acentuação
\usepackage[brazil]{babel}  % suporte para portugues
\usepackage{fancyhdr}       % cabeçalhos e notas de rodapé
\usepackage{graphicx}       % incluir figuras
\usepackage{geometry}       % configurar margens do documento
\usepackage{indentfirst}    % Indenta o primeiro parágrafo de cada seção.
% \usepackage{lmodern}        % fonte Latin Modern 
\usepackage{bigdelim, booktabs, colortbl, longtable, multirow}      % Ferramentas para tabelas
\usepackage{amssymb, amstext, amsthm, icomma}                       % Ferramentas para linguagem matemática
\usepackage{pifont, textcomp, wasysym}                              % Simbolos texto
\usepackage[normalem]{ulem}                                         % Sublinhar texto
\usepackage[alf]{abntex2cite}                                       % Citações padrão ABNT


% -----------------------------------------------------
% comandos adicionais
\newcommand{\cpp}{\texttt{C$++$}}       % escrever C++
\newcommand{\ds}{\displaystyle}         % Tamanho normal das equações
\newcommand{\bsym}[1]{\boldsymbol{#1}}  % Texto no modo matemático em negrito
\newcommand{\mr}[1]{\mathrm{#1}}        % Texto no modo matemático normal (não itálico)
\newcommand{\pare}[1]{\left(#1\right)}  % Parênteses
\newcommand{\colc}[1]{\left[#1\right]}  % Colchetes
\newcommand{\nomeequacao}{Equação}
\newcommand{\nomeequacoes}{Equações}


% -----------------------------------------------------
% espaçamento entre um paragrafo e outro
\setlength{\parskip}{0.2cm}

% % -----------------------------------------------------
% % ajuste de margens no documento 
% \setlrmarginsandblock{3cm}{2cm}{*}
% \setulmarginsandblock{3cm}{2cm}{*}
% \checkandfixthelayout

% -----------------------------------------------------
% informações do tipo de documento
\TipoDeDocumento{Trabalho de Conclusão de Curso}
\NivelDeFormacao{Bacharelado}
\TituloPretendido{Bacharel}

% -----------------------------------------------------
% informações da monografia

% ------- título
% Título do documento em PORTUGUÊS (resumo)
\TituloEmMultiplasLinhas{Proposta de Projeto - \textit{Sir Galahad}}%d

% Título do trabalho em INGLÊS (abstract)
\TituloEmMultiplasLinhasIngles{Project proposal - Sir Galahad}

% ------- autora da monografia
\NomeDoAutor{Guilherme Pires Silva}
\SobrenomeDoAutor{Silva}
\PrenomeDoAutor{Guilherme Pires}

\AtribuiAutorDois{true}%%
\NomeDoAutorDois{Letícia Rodrigues Pinto}
\SobrenomeDoAutorDois{Pinto}
\PrenomeDoAutorDois{Letícia Rodrigues}

\AtribuiAutorTres{true}%%
\NomeDoAutorTres{Raphael Felix}
\SobrenomeDoAutorTres{Felix}
\PrenomeDoAutorTres{Rapahel}

% ------- orientador
\AtribuicaoOrientador{Orientador}%% Atribuição "Orientador(a)"
\TituloDoOrientador{Prof.}%% Título do(a) orientador(a)
\NomeDoOrientador{Daniel Rossato de Oliveira}%% Nome completo do(a) orientador(a)

% ------- coorientador
%% Usado para citação: "\SobrenomeDoCoorientador, PrenomeDoCoorientador" (ex: "Doe, John" ou "Doe, J.")
\AtribuiCoorientador{false}%% Insere ou remove o(a) coorientador(a): "true" ou "false"

% ------ instituição
\Instituicao{Universidade Tecnológica Federal do Paraná}
\Institution{Universidade Tecnológica Federal do Paraná} % [abstract] Institution name (*nome sem traduzir é o recomendado para docs. da UTFPR)
\SiglaInstituicao{UTFPR}

\Departamento{Departamento Acadêmico de Eletrônica}
\SiglaDepartamento{DAELN}
\Curso{Engenharia da Computação, Engenharia Elétrica e Engenharia Mecatrônica}
\Course{Computer Engineering, Electrical Engineering and Mechatronics Engineering}

\Cidade{Curitiba}

\Ano{2023}

% ---------------------------------------------------------------------
% folha de rosto
\DescricaoDoDocumento{
Proposta de projeto apresentado como requisito para aprovação na disciplina Oficina De Integração 2, do \imprimirppgoudepartamento, da \imprimirinstituicao\ (\imprimirsiglainstituicao).
}

% -----------------------------------------------------

\begin{document}

% pre textual ------------------------------------------
% inclui capa do documento
\incluircapa

% formatação da pagina dos elementos pre-textuais
\pretextual

% tipo de licença na que aparece na folha de rosto
\Licenca{CC-BY}

% incluir folha de rosto
\incluirfolhaderosto

% dedicatoria
%%%% Dedicatoria 
%%
%% Texto em que o autor presta homenagem ou dedica seu trabalho.

\begin{dedicatoria}%% Ambiente dedicatoria
    \vspace*{\fill}
    Dedico este trabalho a minha família e aos meus amigos, pelos momentos de ausência.
    \vspace*{\fill}
\end{dedicatoria}

% agradecimentos 
% ------------- AGRADECIMENTOS
%% Texto em que o autor faz agradecimentos dirigidos àqueles que contribuíram de maneira relevante à elaboração do trabalho.

\begin{agradecimentos}[Agradecimentos]%% (opção: título do agradecimento)

    Espaço destinado aos agradecimentos (elemento opcional). Folha que contém manifestação de
    reconhecimento a pessoas e/ou instituições que realmente contribuíram com o(a) autor(a), devendo
    ser expressos de maneira simples. Exemplo:
    
    Não devem ser incluídas informações que nominem empresas ou instituições não nominadas no
    trabalho.
    
    Se o aluno recebeu bolsa de fomento à pesquisa, informar o nome completo da agência de fomento.
    Ex: Capes, CNPq, Fundação Araucária, UTFPR, etc. Incluir o número do projeto após a agência de
    fomento. Este item deve ser o último.
    
    Atenção: não utilizar este exemplo na versão final. Use a sua criatividade!

\end{agradecimentos}
    

% epigrafe
%%%% EPÍGRAFE
%%
%% Texto em que o autor apresenta uma citação, seguida de indicação de autoria, relacionada com a matéria tratada no corpo do
%% trabalho.

\begin{epigrafe}%% Ambiente epigrafe

    \vspace*{\fill}
    \begin{flushright}

        Espaço destinado à epígrafe (elemento opcional). Nesta folha, o(a) autor(a) usa uma citação, seguida
        de indicação de autoria e ano, relacionada, preferencialmente, com o assunto tratado no corpo do
        trabalho. A citação deverá constar na lista de referências.
        
    \end{flushright}

\end{epigrafe}
    

% resumo
% resumo

\begin{resumo}% Ambiente resumo

    
    O resumo deve ser redigido na terceira pessoa do singular, com verbo na voz ativa, não ultrapassando uma página (de 150 a 500 palavras, segundo a ABNT NBR 6028), evitando-se o uso de parágrafos no meio do resumo, assim como fórmulas, equações e símbolos. Iniciar o resumo situando o trabalho no contexto geral, apresentar os objetivos, descrever a metodologia adotada, relatar a contribuição própria, comentar os resultados obtidos e finalmente apresentar as conclusões mais importantes do trabalho. As palavras-chave devem aparecer logo abaixo do resumo, antecedidas da expressão Palavras-chave. Para definição das palavras-chave (e suas correspondentes em inglês no \textit{abstract}) consultar em Termo tópico do Catálogo de Autoridades da Biblioteca Nacional, disponível em: \url{http://acervo.bn.br/sophia_web/index.html}.

    %\vspace*{\offinterlineskip} % pular uma linha
    \noindent
    \textbf{Palavras-chaves}: palavra 1. palavra 2, palavra 3

\end{resumo}
    

% abstract
% Abstract

\begin{resumo}[Abstract]% Ambiente abstract

    \begin{otherlanguage}{english}
        
        Translation of the abstract into English
    
        %\vspace*{\offinterlineskip} % pular uma linha
        \noindent
        \textbf{Keywords}: word 1; word 2; word 3
    
    \end{otherlanguage}

\end{resumo}
    

% lista de ilustrações
\incluirlistadeilustracoes

% lista de quadros e tabelas
\incluirlistadetabelas

% lista de abreviaturas e siglas
% lista de apreviaturas

\begin{siglas}
    \item[AGV] \textit{Automated Guided Vehicle}, traduzido, veículo autoguiado
    \item[AMR] \textit{Autonomous Mobile Robot}, traduzido, robô móvel autônomo
    \item[EUA] Estados Unidos da América
    \item[ROS] \textit{Robot Operating System}, traduzido, sistema operacional robótico
    \item[SLAM] \textit{Simultaneous Localization And Mapping}, traduzido localização e mapeamento simultâneos
    
\end{siglas}
    

% lista de simbolos
% lista de simbolos

\begin{simbolos}
    \item[$ \Gamma $] Letra grega Gama
    \item[$ \Lambda $] Lambda
    \item[$ \zeta $] Letra grega minúscula zeta
    \item[$ \in $] Pertence
\end{simbolos}
    

% incluir sumário
\incluirsumario 

% textual ------------------------------------------------
\textual

% capítulo 1 - introdução
% capítulo 1 - introdução
% ------------------------------
% estrutura do capítulo
% 1. Introdução
% 1.1 Considerações iniciais
% 1.2 Objetivos
%     1.2.1 Objetivo geral 
% 1.3 Justificativa
% 1.4 Declaração do escopo de alto nível
%     1.4.1 Requisitos funcionais
%     1.4.2 Requisitos não-funcionais
% 1.5 Materiais e Métodos
% 1.6 Integração
% 1.7 Análise de Riscos
% 1.8 Estrutura do trabalho
% 1.9 Cronograma
% -------------------------------

% 1 ------------- introdução
\chapter{Introdução}\label{cap:introducao}

Nesta seção introdutória, apresentamos o projeto \textit{Sir Galahad}, um robô equilibrista de duas rodas uma ideia que une a robótica e o controle de sistemas dinâmicos. Nosso enfoque reside em desenvolver um robô capaz de manter seu equilíbrio enquanto se movimenta em ambientes diversos. Ao longo deste documento, examinaremos os objetivos, justificativas e escopo do projeto, além de destacar a importância desse tipo de tecnologia na busca por soluções autônomas e versáteis.

% 1.1 ----------- considerações iniciais
\section{\textbf{Considerações iniciais}}

Nesta seção, exploramos o contexto que motiva a criação do projeto \textit{Sir Galahad}. A busca por um robô equilibrista de duas rodas, capaz de buscar e atingir um alvo, de forma que este trabalho combine conceitos, como de controle dinâmico, fusão de sensores e integração de sistemas. 

A criação de um robô que equilibra-se sobre duas rodas, de forma análoga a um pêndulo invertido, desafia os paradigmas convencionais da mobilidade robótica. Esta iniciativa alinha-se à busca por soluções tecnológicas que, além de atender necessidades práticas, inspiram a exploração dos limites da engenhosidade humana. Neste contexto, a motivação diante da escolha do projeto  \textit{Sir Galahad} é permeada por questões técnicas, científicas e criativas. 

Este trabalho se propõe a detalhar o projeto, desde suas premissas até sua concretização. Ao explorar as considerações iniciais que permeiam \textit{Sir Galahad}. 

% 1.2 ----------- objetivos
\section{\textbf{Objetivos}}
Os objetivos deste projeto refletem uma busca coerente por resultados concretos que aliam a funcionalidade técnica com as expectativas criativas. Compreender e atender esses objetivos é fundamental para avaliar o sucesso do projeto proposto.

% 1.2.1 ----------- objetivo geral
\subsection{Objetivo geral}
% no maximo 3 linhas 
O objetivo geral deste trabalho é o projeto e implementação de um robô diferencial de duas rodas, equilibrista, que deve ser capaz de procurar e identificar um alvo (previamente definido),  combinando controle dinâmico, sensoriamento e integração de sistemas.

% 1.3 ----------- justificativa
\section{\textbf{Justificativa}}

A criação do projeto \textit{Sir Galahad} é motivada pela busca por avanços na robótica e suas aplicações práticas. Ao  desenvolver um equilibrista de duas rodas, podemos explorar o potencial da robótica em situações desafiantes, contribuindo para o conhecimento de controle e sensoriamento.

A capacidade do \textit{Sir Galahad} de buscar, identificar e atingir alvos tem implicações práticas em inspeção e monitoramento, além de ser uma expressão da criatividade humana na tecnologia robótica.

% 1.4 ----------- declaração do escopo de alto nível
\section{\textbf{Declaração do escopo de alto nível}}

O protótipo criado no projeto consiste em um veículo autônomo que deve detectar um alvo ou objetivo por meio de reconhecimento de imagem e se locomover em direção à ele enquanto se equilibra sobre duas rodas como um pêndulo invertido. 

% 1.4.1 ----------- requisitos funcionais
\subsection{Requisitos funcionais}

RF1: O robô deve ser capaz de se manter em equilíbrio sobre duas rodas estando parado ou em movimento em uma superfície lisa em um ambiente fechado.

RF2: O robô deve ser capaz de detectar por reconhecimento de imagem uma bola de tênis em um ambiente bem iluminado.

RF3: O veículo deve ser capaz de se locomover em direção ao alvo detectado contanto que o terreno seja regular e não existam obstáculos em seu caminho.

RF4: O robô deve ser capaz de sinalizar que chegou perto ou atingiu o alvo.


% 1.4.2 ----------- requisitos não-funcionais
\subsection{Requisitos não-funcionais}


% 1.5 ----------- materiais e métodos
\section{\textbf{Materiais e métodos}}
Nesta seção, descrevemos os materiais e custos esperados, além da abordagem metodológica adotada para o desenvolvimento do robô equilibrista.

\textbf{Materiais:}
\begin{itemize}
    \item 1x Raspberry Pi 3 Model B 
    \item 1x ESP32
    \item 
\end{itemize}

% 1.6 ----------- integração
\section{\textbf{Integração}}


% 1.7 ----------- analise de riscos 
\section{\textbf{Análise de riscos}}


% 1.8 ----------- estrutura do trabalho
\section{\textbf{Estrutura do trabalho}}


% 1.9 ----------- cronograma 
\section{\textbf{Cronograma}}

% capitulo 2 - revisão de literatura
% capitulo 2 - revisão de literatura
% ------------------------------------
% estrutura: 
    % 2. Revisão de literatura
    % 2.1. Conceitos de robótica móvel
    %     2.1.1 Sistema de locomoção
    %         2.1.1.1 Atuadores
    %         2.1.1.2 Cinemática
    %     2.1.2 Sistema de percepção
    %         2.1.2.1 Tipo de sensores
    %         2.1.2.2 Características de erros e imprecisão do sistema
    %     2.1.3 Sistema de localização
    %         2.1.3.1 Odometria
    %         2.1.3.2 Mapeamento
    %         2.1.3.3 Navegação
    %     2.1.4 Sistema de controle
    % 2.2. Aspectos de um AMR
    %     2.2.1 Caracteristicas mecanicas
    %     2.2.2 Sistemas embarcados
    %     2.2.3 Comunicação e integração de sensores e atuadores
    %         2.2.3.1Filtros
    %     2.2.4 Controle 
    %     2.2.5 Navegação
    %     2.2.6. ROS – Robotic operating system
    %     2.2.7. Competições Robomagellan como exemplo de aplicação de AMR
    % 2.3. Estado da arte
    % 2.4. Metodologia de pesquisa

    % 2. ------------------ revisão de literatura 
    \chapter{Revisão de literatura}\label{cap:revisão de literatura}

    % 2.1 ----------------- conceitos de robórica móvel
    \section{\textbf{Conceitos de robótica móvel}}

    % 2.1.1 --------------- sistema de locomoção
    \subsection{Sistema de Locomoção}

    % 2.1.1.1 ------------- atuadores
    \subsubsection{\uline{Atuadores}}

    % 2.1.1.2 ------------- cinematica
    \subsubsection{\uline{Cinemática}}

    % 2.1.2  -------------- sistema de percepção
    \subsection{Sistema de percepção}

    % 2.1.2.1 ------------- tipos de sensores 
    \subsubsection{\uline{Tipo de sensores}}

    % 2.1.2.2 ------------- características de erros e imprecisão do sistema
    \subsubsection{\uline{Características de erros e imprecisão do sistema}}

    % 2.1.3 --------------- sistema de localização
    \subsection{Sistema de localização}

    % 2.1.3.1 ------------- odometria 
    \subsubsection{\uline{Odometria}}

    % 2.1.3.2 ------------- mapeamento
    \subsubsection{\uline{Mapeamento}}

    % 2.1.3.3 ------------- navegação
    \subsubsection{\uline{Navegação}}

    % 2.1.4 --------------- sistema de controle
    \subsection{Sistema de Controle}

    % 2.2 ---------------- aspectos de um AMR
    \section{\textbf{Aspectos de um AMR}}

    % 2.2.1 -------------- caracteristicas mecanicas
    \subsection{Características mecânicas}

    % 2.2.2 --------------- sistemas embarcados
    \subsection{Sistemas embarcados}

    % 2.2.3 --------------- comunicação e integração de sensores e atuadores
    \subsection{Comunicação e integração de sensores e atuadores}

    % 2.2.3.1 -------------- filtros
    \subsubsection{\uline{Filtros}}

    % 2.2.4 ---------------- controle
    \subsection{Controle}

    % 2.2.5 ---------------- navegação
    \subsection{Navegação}

    % 2.2.6 ------------------ ROS – Robotic operating system
    \subsection{ROS - Robot operating system}

    % 2.2.7 ------------------ competições Robomagellan como exemplo de aplicação de AMR
    \subsection{Competições Robomagellan como exemplo de aplicação de AMR}

    % 2.3 -------------------- estado da arte 
    \section{\textbf{Estado da arte}}

    % 2.4 -------------------- metodologia de pesquisa
    \section{\textbf{Metodologia de pesquisa}}



% capitulo 3 - design e arquitetura do robo
% capitulo 3 - Design e arquitetura do robô
% ------------------------------------
% estrutura: 
% 3. Design e arquitetura do robô
%     3.1. Projeto mecânico e esquema de montagem
%     3.2. Projeto de hardware
%         3.2.1. Diagrama elétrico para circuito digital
%         3.2.2. Diagrama elétrico para circuito de potência
%     3.3. Projeto de software
%         3.3.1. Sistema embarcado
%         3.3.2. Sistema de percepção
%         3.3.3. Sistema de controle
%         3.3.4. Localização e controle de movimento
%         3.3.5. Mapeamento
%         3.3.6. Planejamento de rotas
%     3.4. Integração de sistemas via ROS
%     3.5. Arquitetura de software

    % 3. ------------------ design e arquitetura do robô 
    \chapter{Design e arquitetura do robô}\label{cap: Design e arquitetura do robô}

    % 3.1 ----------------- projeto mecanico e esquema de montagem 
    \section{\textbf{Projeto mecânico e esquema de montagem}}


    % 3.2 ----------------- projeto de hardware
    \section{\textbf{Projeto de hardware}}

    % 3.2.1 -------------- diagrama eletrico para circuito digital 
    \subsection{Diagrama elétrico para circuito digital}

    % 3.2.2 -------------- digrama eletrico para circuito de potencia
    \subsection{Diagrama elétrico para circuito de potência}

    % 3.3 ---------------- projeto de software
    \section{\textbf{Projeto de software}}

    % 3.3.1 --------------- sistema embarcado
    \subsection{Sistema embarcado}

    % 3.3.2 --------------- sistema de percepção
    \subsection{Sistema de percepção}

    % 3.3.3 --------------- sistema de controle 
    \subsection{Sistema de controle}

    % 3.3.4 --------------- localização e controle de movimento
    \subsection{Localização e controle de movimento}

    % 3.3.5 --------------- mapeamento
    \subsection{Mapeamento}

    % 3.3.6 --------------- planejamento de rotas
    \subsection{Planejamento de rotas}
    
    % 3.4 ------------------ integração de sistemas via ROS
    \section{\textbf{Integração de sistemas via ROS}}

    % 3.5 ------------------ arquitetura de software
    \section{\textbf{Arquitetura de software}}

    








% capitulo 4 - implementação e montagem do robo
% capitulo 4 - implementação e montagem do robo
% ------------------------------------
% estrutura: 
% 4. Implementação e montagem do robô
%     4.1. Montagem da estrutura mecânica
%     4.2. Implementação e montagem de componentes eletrônicos e atuadores
%         4.2.1. PCB
%         4.2.2. Montagem eletrônica
%     4.3. Implementação de software
%         4.3.1. Configuração de parâmetros de utilização
%     4.4. Validação


    % 4. ------------------ implementação e montagem do robô 
    \chapter{Implementação e montagem do robô}\label{cap: Implementação e montagem do robô}

    % 4.1 ----------------- montagem da estrutura mecânica
    \section{\textbf{Montagem da estrutura mecânica}}

    % 4.2 ----------------- implementação e montagem de componentes eletrônicos e atuadores
    \section{\textbf{Implementação e montagem de componentes eletrônicos e atuadores}}

    % 4.2.1 --------------- pcb
    \subsection{PCB}

    % 4.2.2 --------------- montagem eletrônica
    \subsection{Montagem eletrônica}

    % 4.3 ----------------- implementação de software
    \section{\textbf{Implementação de software}}

    % 4.3.1 --------------- configuração de parametros de utilização
    \subsection{Configuração de parâmetros de utilização}

    % 4.1 ----------------- validação
    \section{\textbf{Validação}}



% capitulo 5 - testes e resultados
% capitulo 5 - testes e resultados
% ------------------------------------
% estrutura: 
% 5. Testes e resultados
%   5.1 Testes em sistemas isolados
%       5.1.1 Sistema embarcado
%       5.1.2 Locomoção
%       5.1.3 Sistema de percepção
%       5.1.4 Localização
%       5.1.5 Mapeamento
%   5.2 Teste de desempenho do robo
%       5.2.1 Precisão de localização
%       5.2.2 Desempenho de navegação 
%       5.2.3 Confiabilidade e robustez 
%   5.3 Desempenho em competições
%       5.3.1 Condições de prova
%       5.3.2 Resultados
%       5.3.3 Analise de resultados

% 5. ----------------------------- testes e resultados
\chapter{Testes e resultados}\label{cap: Testes e resultados}

% 5.1 ---------------------------- testes em sistemas isolados
\section{\textbf{Testes em sistemas isolados}}

% 5.1.1 -------------------------- sistema embarcado
\subsection{Sistema embarcado}

    \subsection*{\textbf{Condições de teste:}}

    \subsection*{\textbf{Resultados:}}  

    \subsection*{\textbf{Análise de resultados:}}

% 5.1.2 -------------------------- locomoção
\subsection{Locomoção}

\subsection*{\textbf{Condições de teste:}}

\subsection*{\textbf{Resultados:}}  

\subsection*{\textbf{Análise de resultados:}}


% 5.1.3 -------------------------- sistema de percepção
\subsection{Sistema de percepção}

\subsection*{\textbf{Condições de teste:}}

\subsection*{\textbf{Resultados:}}  

\subsection*{\textbf{Análise de resultados:}}


% 5.1.4 -------------------------- localização
\subsection{Localização}

\subsection*{\textbf{Condições de teste:}}

\subsection*{\textbf{Resultados:}}  

\subsection*{\textbf{Análise de resultados:}}


% 5.1.5 -------------------------- mapeamento
\subsection{Mapeamento}

\subsection*{\textbf{Condições de teste:}}

\subsection*{\textbf{Resultados:}}  

\subsection*{\textbf{Análise de resultados:}}


% 5.2 ---------------------------- testes de desempenho
\section{\textbf{Testes de desempenho}}

% 5.2.1 -------------------------- precisão de localização
\subsection{Precisão de localização}

\subsection*{\textbf{Condições de teste:}}

\subsection*{\textbf{Resultados:}}  

\subsection*{\textbf{Análise de resultados:}}


% 5.2.2 -------------------------- desempenho de navegação
\subsection{Desempenho em navegação}

\subsection*{\textbf{Condições de teste:}}

\subsection*{\textbf{Resultados:}}  

\subsection*{\textbf{Análise de resultados:}}


% 5.2.3 -------------------------- confiabilidade e robustez
\subsection{Confiabilidade e robustez}

\subsection*{\textbf{Condições de teste:}}

\subsection*{\textbf{Resultados:}}  

\subsection*{\textbf{Análise de resultados:}}


% 5.3 ---------------------------- desempenho em competições
\section{\textbf{Desempenho em competições}}

% 5.3.1 -------------------------- condição de prova
\subsection{Condição de prova}

% 5.3.2 -------------------------- resultados
\subsection{Resultados}

% 5.3.3 -------------------------- analise de resultados
\subsection{Análise de resultados}

% capitulo 6 - conclusão
% capitulo 6 - conclusão
% ------------------------------------
% estrutura: 
% 6. Conclusão
%     6.1. Principais resultados e contribuições
%     6.2. Limitações e trabalhos futuros
%     6.3. Considerações finais

% 6. ----------------------------- conclusão
\chapter{Conclusão}\label{cap: Conclusão}

% 6.1 ---------------------------- principais resultados e contribuições
\section{\textbf{Principais resultados e contribuições}}

% 6.2 ---------------------------- limitações e trabalhos futuros
\section{\textbf{Limitações e trabalhos futuros}}

% 6.3 ---------------------------- considerações finais
\section{\textbf{Considerações finais}}

\arquivosdereferencias{posTextual/Referencias/library}
\end{document}