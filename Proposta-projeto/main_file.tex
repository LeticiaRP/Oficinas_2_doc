% definição tipo de documento
\documentclass[
12pt,       % tamanho da fonte
openright,  % inicio documento a direita
oneside,    % frente e verso
a4paper,    % folha a4
sumario = abnt-6027-2012, % Formatação do sumário: tradicional (estilo tradicional) ou abnt-6027-2012 (norma ABNT 6027-2012)
chapter = TITLE,          % Títulos de capítulos em maiúsculas 
%section = TITLE,
pretextualoneside,        % Impressão dos elementos pré-textuais pretextualoneside em um lado da folha
fontetimes,               % Fonte do texto: (times)
% fontearial,                % Fonte do texto: (arial) 
% fontecourier            % Fonte do texto: (courier), 
% fontemodern,              % Fonte do texto: (lmodern - default latex)
semrecuonosumario,        % Remoção do recuo dos itens no sumário (comente para adição do recuo, se estilo tradicional)
legendascentralizadas,    % Alinhamento das legendas centralizado (comente para alinhamento à esquerda)
pardeassinaturas,         % Assinaturas na folha de aprovação em até duas colunas (comente para em uma única coluna)
linhasdeassinaturas,      % Linhas de assinaturas na folha de aprovação (comente para remover as linhas)
english,french,spanish,
brazil      % o ultimo idioma é o padrão do documento
]{utfprct}
% -----------------------------------------------------
% pacotes utilizados no documento
\usepackage[utf8]{inputenc} % codificação texto
\usepackage[T1]{fontenc}    % codificação para adicionar acentuação
\usepackage[brazil]{babel}  % suporte para portugues
\usepackage{fancyhdr}       % cabeçalhos e notas de rodapé
\usepackage{graphicx}       % incluir figuras
\usepackage{geometry}       % configurar margens do documento
\usepackage{indentfirst}    % Indenta o primeiro parágrafo de cada seção.
% \usepackage{lmodern}        % fonte Latin Modern 
\usepackage{bigdelim, booktabs, colortbl, longtable, multirow}      % Ferramentas para tabelas
\usepackage{amssymb, amstext, amsthm, icomma}                       % Ferramentas para linguagem matemática
\usepackage{pifont, textcomp, wasysym}                              % Simbolos texto
\usepackage[normalem]{ulem}                                         % Sublinhar texto
\usepackage[alf]{abntex2cite}                                       % Citações padrão ABNT


% -----------------------------------------------------
% comandos adicionais
\newcommand{\cpp}{\texttt{C$++$}}       % escrever C++
\newcommand{\ds}{\displaystyle}         % Tamanho normal das equações
\newcommand{\bsym}[1]{\boldsymbol{#1}}  % Texto no modo matemático em negrito
\newcommand{\mr}[1]{\mathrm{#1}}        % Texto no modo matemático normal (não itálico)
\newcommand{\pare}[1]{\left(#1\right)}  % Parênteses
\newcommand{\colc}[1]{\left[#1\right]}  % Colchetes
\newcommand{\nomeequacao}{Equação}
\newcommand{\nomeequacoes}{Equações}


% -----------------------------------------------------
% espaçamento entre um paragrafo e outro
\setlength{\parskip}{0.2cm}

% % -----------------------------------------------------
% % ajuste de margens no documento 
% \setlrmarginsandblock{3cm}{2cm}{*}
% \setulmarginsandblock{3cm}{2cm}{*}
% \checkandfixthelayout

% -----------------------------------------------------
% informações do tipo de documento
\TipoDeDocumento{Trabalho de Conclusão de Curso}
\NivelDeFormacao{Bacharelado}
\TituloPretendido{Bacharel}

% -----------------------------------------------------
% informações da monografia

% ------- título
% Título do documento em PORTUGUÊS (resumo)
\TituloEmMultiplasLinhas{Proposta de Projeto - \textit{Sir Galahad}}%d

% Título do trabalho em INGLÊS (abstract)
\TituloEmMultiplasLinhasIngles{Project proposal - Sir Galahad}

% ------- autora da monografia
\NomeDoAutor{Guilherme Pires Silva}
\SobrenomeDoAutor{Silva}
\PrenomeDoAutor{Guilherme Pires}

\AtribuiAutorDois{true}%%
\NomeDoAutorDois{Letícia Rodrigues Pinto}
\SobrenomeDoAutorDois{Pinto}
\PrenomeDoAutorDois{Letícia Rodrigues}

\AtribuiAutorTres{true}%%
\NomeDoAutorTres{Raphael Felix}
\SobrenomeDoAutorTres{Felix}
\PrenomeDoAutorTres{Rapahel}

% ------- orientador
\AtribuicaoOrientador{Orientador}%% Atribuição "Orientador(a)"
\TituloDoOrientador{Prof.}%% Título do(a) orientador(a)
\NomeDoOrientador{Daniel Rossato de Oliveira}%% Nome completo do(a) orientador(a)

% ------- coorientador
%% Usado para citação: "\SobrenomeDoCoorientador, PrenomeDoCoorientador" (ex: "Doe, John" ou "Doe, J.")
\AtribuiCoorientador{false}%% Insere ou remove o(a) coorientador(a): "true" ou "false"

% ------ instituição
\Instituicao{Universidade Tecnológica Federal do Paraná}
\Institution{Universidade Tecnológica Federal do Paraná} % [abstract] Institution name (*nome sem traduzir é o recomendado para docs. da UTFPR)
\SiglaInstituicao{UTFPR}

\Departamento{Departamento Acadêmico de Eletrônica}
\SiglaDepartamento{DAELN}
\Curso{Engenharia da Computação, Engenharia Elétrica e Engenharia Mecatrônica}
\Course{Computer Engineering, Electrical Engineering and Mechatronics Engineering}

\Cidade{Curitiba}

\Ano{2023}

% ---------------------------------------------------------------------
% folha de rosto
\DescricaoDoDocumento{
Proposta de projeto apresentado como requisito para aprovação na disciplina Oficina De Integração 2, do \imprimirppgoudepartamento, da \imprimirinstituicao\ (\imprimirsiglainstituicao).
}

% -----------------------------------------------------

\begin{document}

% pre textual ------------------------------------------
% inclui capa do documento
\incluircapa

% formatação da pagina dos elementos pre-textuais
\pretextual

% tipo de licença na que aparece na folha de rosto
\Licenca{CC-BY}

% incluir folha de rosto
\incluirfolhaderosto

% resumo
% resumo

\begin{resumo}% Ambiente resumo

    
    O resumo deve ser redigido na terceira pessoa do singular, com verbo na voz ativa, não ultrapassando uma página (de 150 a 500 palavras, segundo a ABNT NBR 6028), evitando-se o uso de parágrafos no meio do resumo, assim como fórmulas, equações e símbolos. Iniciar o resumo situando o trabalho no contexto geral, apresentar os objetivos, descrever a metodologia adotada, relatar a contribuição própria, comentar os resultados obtidos e finalmente apresentar as conclusões mais importantes do trabalho. As palavras-chave devem aparecer logo abaixo do resumo, antecedidas da expressão Palavras-chave. Para definição das palavras-chave (e suas correspondentes em inglês no \textit{abstract}) consultar em Termo tópico do Catálogo de Autoridades da Biblioteca Nacional, disponível em: \url{http://acervo.bn.br/sophia_web/index.html}.

    %\vspace*{\offinterlineskip} % pular uma linha
    \noindent
    \textbf{Palavras-chaves}: palavra 1. palavra 2, palavra 3

\end{resumo}
    

% abstract
% Abstract

\begin{resumo}[Abstract]% Ambiente abstract

    \begin{otherlanguage}{english}
        
        Translation of the abstract into English
    
        %\vspace*{\offinterlineskip} % pular uma linha
        \noindent
        \textbf{Keywords}: word 1; word 2; word 3
    
    \end{otherlanguage}

\end{resumo}
    

% lista de ilustrações
\incluirlistadeilustracoes

% incluir sumário
\incluirsumario 

% textual ------------------------------------------------
\textual

% capítulo 1 - introdução
% capítulo 1 - introdução
% ------------------------------
% estrutura do capítulo
% 1. Introdução
% 1.1 Considerações iniciais
% 1.2 Objetivos
%     1.2.1 Objetivo geral 
% 1.3 Justificativa
% 1.4 Declaração do escopo de alto nível
%     1.4.1 Requisitos funcionais
%     1.4.2 Requisitos não-funcionais
% 1.5 Materiais e Métodos
% 1.6 Integração
% 1.7 Análise de Riscos
% 1.8 Estrutura do trabalho
% 1.9 Cronograma
% -------------------------------

% 1 ------------- introdução
\chapter{Introdução}\label{cap:introducao}

Nesta seção introdutória, apresentamos o projeto \textit{Sir Galahad}, um robô equilibrista de duas rodas uma ideia que une a robótica e o controle de sistemas dinâmicos. Nosso enfoque reside em desenvolver um robô capaz de manter seu equilíbrio enquanto se movimenta em ambientes diversos. Ao longo deste documento, examinaremos os objetivos, justificativas e escopo do projeto, além de destacar a importância desse tipo de tecnologia na busca por soluções autônomas e versáteis.

% 1.1 ----------- considerações iniciais
\section{\textbf{Considerações iniciais}}

Nesta seção, exploramos o contexto que motiva a criação do projeto \textit{Sir Galahad}. A busca por um robô equilibrista de duas rodas, capaz de buscar e atingir um alvo, de forma que este trabalho combine conceitos, como de controle dinâmico, fusão de sensores e integração de sistemas. 

A criação de um robô que equilibra-se sobre duas rodas, de forma análoga a um pêndulo invertido, desafia os paradigmas convencionais da mobilidade robótica. Esta iniciativa alinha-se à busca por soluções tecnológicas que, além de atender necessidades práticas, inspiram a exploração dos limites da engenhosidade humana. Neste contexto, a motivação diante da escolha do projeto  \textit{Sir Galahad} é permeada por questões técnicas, científicas e criativas. 

Este trabalho se propõe a detalhar o projeto, desde suas premissas até sua concretização. Ao explorar as considerações iniciais que permeiam \textit{Sir Galahad}. 

% 1.2 ----------- objetivos
\section{\textbf{Objetivos}}
Os objetivos deste projeto refletem uma busca coerente por resultados concretos que aliam a funcionalidade técnica com as expectativas criativas. Compreender e atender esses objetivos é fundamental para avaliar o sucesso do projeto proposto.

% 1.2.1 ----------- objetivo geral
\subsection{Objetivo geral}
% no maximo 3 linhas 
O objetivo geral deste trabalho é o projeto e implementação de um robô diferencial de duas rodas, equilibrista, que deve ser capaz de procurar e identificar um alvo (previamente definido),  combinando controle dinâmico, sensoriamento e integração de sistemas.

% 1.3 ----------- justificativa
\section{\textbf{Justificativa}}

A criação do projeto \textit{Sir Galahad} é motivada pela busca por avanços na robótica e suas aplicações práticas. Ao  desenvolver um equilibrista de duas rodas, podemos explorar o potencial da robótica em situações desafiantes, contribuindo para o conhecimento de controle e sensoriamento.

A capacidade do \textit{Sir Galahad} de buscar, identificar e atingir alvos tem implicações práticas em inspeção e monitoramento, além de ser uma expressão da criatividade humana na tecnologia robótica.

% 1.4 ----------- declaração do escopo de alto nível
\section{\textbf{Declaração do escopo de alto nível}}

O protótipo criado no projeto consiste em um veículo autônomo que deve detectar um alvo ou objetivo por meio de reconhecimento de imagem e se locomover em direção à ele enquanto se equilibra sobre duas rodas como um pêndulo invertido. 

% 1.4.1 ----------- requisitos funcionais
\subsection{Requisitos funcionais}

RF1: O robô deve ser capaz de se manter em equilíbrio sobre duas rodas estando parado ou em movimento em uma superfície lisa em um ambiente fechado.

RF2: O robô deve ser capaz de detectar por reconhecimento de imagem uma bola de tênis em um ambiente bem iluminado.

RF3: O veículo deve ser capaz de se locomover em direção ao alvo detectado contanto que o terreno seja regular e não existam obstáculos em seu caminho.

RF4: O robô deve ser capaz de sinalizar que chegou perto ou atingiu o alvo.


% 1.4.2 ----------- requisitos não-funcionais
\subsection{Requisitos não-funcionais}


% 1.5 ----------- materiais e métodos
\section{\textbf{Materiais e métodos}}
Nesta seção, descrevemos os materiais e custos esperados, além da abordagem metodológica adotada para o desenvolvimento do robô equilibrista.

\textbf{Materiais:}
\begin{itemize}
    \item 1x Raspberry Pi 3 Model B 
    \item 1x ESP32
    \item 
\end{itemize}

% 1.6 ----------- integração
\section{\textbf{Integração}}


% 1.7 ----------- analise de riscos 
\section{\textbf{Análise de riscos}}


% 1.8 ----------- estrutura do trabalho
\section{\textbf{Estrutura do trabalho}}


% 1.9 ----------- cronograma 
\section{\textbf{Cronograma}}

\end{document}