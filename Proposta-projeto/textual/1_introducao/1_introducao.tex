% capítulo 1 - introdução
% ------------------------------
% estrutura do capítulo
% 1. Introdução
% 1.1 Considerações iniciais
% 1.2 Objetivos
%     1.2.1 Objetivo geral 
% 1.3 Justificativa
% 1.4 Declaração do escopo de alto nível
%     1.4.1 Requisitos funcionais
%     1.4.2 Requisitos não-funcionais
% 1.5 Materiais e Métodos
% 1.6 Integração
% 1.7 Análise de Riscos
% 1.8 Estrutura do trabalho
% 1.9 Cronograma
% -------------------------------

% 1 ------------- introdução
\chapter{Introdução}\label{cap:introducao}

Nesta seção introdutória, apresentamos o projeto \textit{Sir Galahad}, um robô equilibrista de duas rodas uma ideia que une a robótica e o controle de sistemas dinâmicos. Nosso enfoque reside em desenvolver um robô capaz de manter seu equilíbrio enquanto se movimenta em ambientes diversos. Ao longo deste documento, examinaremos os objetivos, justificativas e escopo do projeto, além de destacar a importância desse tipo de tecnologia na busca por soluções autônomas e versáteis.

% 1.1 ----------- considerações iniciais
\section{\textbf{Considerações iniciais}}

Nesta seção, exploramos o contexto que motiva a criação do projeto \textit{Sir Galahad}. A busca por um robô equilibrista de duas rodas, capaz de buscar e atingir um alvo, de forma que este trabalho combine conceitos, como de controle dinâmico, fusão de sensores e integração de sistemas. 

A criação de um robô que equilibra-se sobre duas rodas, de forma análoga a um pêndulo invertido, desafia os paradigmas convencionais da mobilidade robótica. Esta iniciativa alinha-se à busca por soluções tecnológicas que, além de atender necessidades práticas, inspiram a exploração dos limites da engenhosidade humana. Neste contexto, a motivação diante da escolha do projeto  \textit{Sir Galahad} é permeada por questões técnicas, científicas e criativas. 

Este trabalho se propõe a detalhar o projeto, desde suas premissas até sua concretização. Ao explorar as considerações iniciais que permeiam \textit{Sir Galahad}. 

% 1.2 ----------- objetivos
\section{\textbf{Objetivos}}
Os objetivos deste projeto refletem uma busca coerente por resultados concretos que aliam a funcionalidade técnica com as expectativas criativas. Compreender e atender esses objetivos é fundamental para avaliar o sucesso do projeto proposto.

% 1.2.1 ----------- objetivo geral
\subsection{Objetivo geral}
% no maximo 3 linhas 
O objetivo geral deste trabalho é o projeto e implementação de um robô diferencial de duas rodas, equilibrista, que deve ser capaz de procurar e identificar um alvo (previamente definido),  combinando controle dinâmico, sensoriamento e integração de sistemas.

% 1.3 ----------- justificativa
\section{\textbf{Justificativa}}

A criação do projeto \textit{Sir Galahad} é motivada pela busca por avanços na robótica e suas aplicações práticas. Ao  desenvolver um equilibrista de duas rodas, podemos explorar o potencial da robótica em situações desafiantes, contribuindo para o conhecimento de controle e sensoriamento.

A capacidade do \textit{Sir Galahad} de buscar, identificar e atingir alvos tem implicações práticas em inspeção e monitoramento, além de ser uma expressão da criatividade humana na tecnologia robótica.

% 1.4 ----------- declaração do escopo de alto nível
\section{\textbf{Declaração do escopo de alto nível}}

O protótipo criado no projeto consiste em um veículo autônomo que deve detectar um alvo ou objetivo por meio de reconhecimento de imagem e se locomover em direção à ele enquanto se equilibra sobre duas rodas como um pêndulo invertido. 

% 1.4.1 ----------- requisitos funcionais
\subsection{Requisitos funcionais}

RF1: O robô deve ser capaz de se manter em equilíbrio sobre duas rodas estando parado ou em movimento em uma superfície lisa em um ambiente fechado.

RF2: O robô deve ser capaz de detectar por reconhecimento de imagem uma bola de tênis em um ambiente bem iluminado.

RF3: O veículo deve ser capaz de se locomover em direção ao alvo detectado contanto que o terreno seja regular e não existam obstáculos em seu caminho.

RF4: O robô deve ser capaz de sinalizar que chegou perto ou atingiu o alvo.


% 1.4.2 ----------- requisitos não-funcionais
\subsection{Requisitos não-funcionais}


% 1.5 ----------- materiais e métodos
\section{\textbf{Materiais e métodos}}
Nesta seção, descrevemos os materiais e custos esperados, além da abordagem metodológica adotada para o desenvolvimento do robô equilibrista.

\textbf{Materiais:}
\begin{itemize}
    \item 1x Raspberry Pi 3 Model B 
    \item 1x ESP32
    \item 
\end{itemize}

% 1.6 ----------- integração
\section{\textbf{Integração}}


% 1.7 ----------- analise de riscos 
\section{\textbf{Análise de riscos}}


% 1.8 ----------- estrutura do trabalho
\section{\textbf{Estrutura do trabalho}}


% 1.9 ----------- cronograma 
\section{\textbf{Cronograma}}