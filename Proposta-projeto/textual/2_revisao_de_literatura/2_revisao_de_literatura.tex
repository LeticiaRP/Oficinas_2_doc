% capitulo 2 - revisão de literatura
% ------------------------------------
% estrutura: 
    % 2. Revisão de literatura
    % 2.1. Conceitos de robótica móvel
    %     2.1.1 Sistema de locomoção
    %         2.1.1.1 Atuadores
    %         2.1.1.2 Cinemática
    %     2.1.2 Sistema de percepção
    %         2.1.2.1 Tipo de sensores
    %         2.1.2.2 Características de erros e imprecisão do sistema
    %     2.1.3 Sistema de localização
    %         2.1.3.1 Odometria
    %         2.1.3.2 Mapeamento
    %         2.1.3.3 Navegação
    %     2.1.4 Sistema de controle
    % 2.2. Aspectos de um AMR
    %     2.2.1 Caracteristicas mecanicas
    %     2.2.2 Sistemas embarcados
    %     2.2.3 Comunicação e integração de sensores e atuadores
    %         2.2.3.1Filtros
    %     2.2.4 Controle 
    %     2.2.5 Navegação
    %     2.2.6. ROS – Robotic operating system
    %     2.2.7. Competições Robomagellan como exemplo de aplicação de AMR
    % 2.3. Estado da arte
    % 2.4. Metodologia de pesquisa

    % 2. ------------------ revisão de literatura 
    \chapter{Revisão de literatura}\label{cap:revisão de literatura}

    % 2.1 ----------------- conceitos de robórica móvel
    \section{\textbf{Conceitos de robótica móvel}}

    % 2.1.1 --------------- sistema de locomoção
    \subsection{Sistema de Locomoção}

    % 2.1.1.1 ------------- atuadores
    \subsubsection{\uline{Atuadores}}

    % 2.1.1.2 ------------- cinematica
    \subsubsection{\uline{Cinemática}}

    % 2.1.2  -------------- sistema de percepção
    \subsection{Sistema de percepção}

    % 2.1.2.1 ------------- tipos de sensores 
    \subsubsection{\uline{Tipo de sensores}}

    % 2.1.2.2 ------------- características de erros e imprecisão do sistema
    \subsubsection{\uline{Características de erros e imprecisão do sistema}}

    % 2.1.3 --------------- sistema de localização
    \subsection{Sistema de localização}

    % 2.1.3.1 ------------- odometria 
    \subsubsection{\uline{Odometria}}

    % 2.1.3.2 ------------- mapeamento
    \subsubsection{\uline{Mapeamento}}

    % 2.1.3.3 ------------- navegação
    \subsubsection{\uline{Navegação}}

    % 2.1.4 --------------- sistema de controle
    \subsection{Sistema de Controle}

    % 2.2 ---------------- aspectos de um AMR
    \section{\textbf{Aspectos de um AMR}}

    % 2.2.1 -------------- caracteristicas mecanicas
    \subsection{Características mecânicas}

    % 2.2.2 --------------- sistemas embarcados
    \subsection{Sistemas embarcados}

    % 2.2.3 --------------- comunicação e integração de sensores e atuadores
    \subsection{Comunicação e integração de sensores e atuadores}

    % 2.2.3.1 -------------- filtros
    \subsubsection{\uline{Filtros}}

    % 2.2.4 ---------------- controle
    \subsection{Controle}

    % 2.2.5 ---------------- navegação
    \subsection{Navegação}

    % 2.2.6 ------------------ ROS – Robotic operating system
    \subsection{ROS - Robot operating system}

    % 2.2.7 ------------------ competições Robomagellan como exemplo de aplicação de AMR
    \subsection{Competições Robomagellan como exemplo de aplicação de AMR}

    % 2.3 -------------------- estado da arte 
    \section{\textbf{Estado da arte}}

    % 2.4 -------------------- metodologia de pesquisa
    \section{\textbf{Metodologia de pesquisa}}

